%%%%%%%%%%%%%%%%%%%%%%%%%%%%%%%%%%%%%%%%%
% Friggeri Resume/CV
% XeLaTeX Template
% Version 1.0 (5/5/13)
%
% This template has been downloaded from:
% http://www.LaTeXTemplates.com
%
% Original author:
% Adrien Friggeri (adrien@friggeri.net)
% https://github.com/afriggeri/CV
%
% License:
% CC BY-NC-SA 3.0 (http://creativecommons.org/licenses/by-nc-sa/3.0/)
%
% Important notes:
% This template needs to be compiled with XeLaTeX and the bibliography, if used,
% needs to be compiled with biber rather than bibtex.
%
%%%%%%%%%%%%%%%%%%%%%%%%%%%%%%%%%%%%%%%%%

\documentclass[]{friggeri-cv} % Add 'print' as an option into the square bracket to remove colors from this template for printing

\usepackage[backend=biber]{biblatex}
\addbibresource{bibliography.bib}


\begin{document}

\header{Jesper}{Dramsch}{Geophysics Student (MSc)} % Your name and current job title/field

%----------------------------------------------------------------------------------------
%	SIDEBAR SECTION
%----------------------------------------------------------------------------------------

\begin{aside} % In the aside, each new line forces a line break
\section{contact}
Von-Essen-Str. 121
22081 Hamburg
Germany
~
+49 (0) 163 734 2207
~
\href{mailto:jesper@dramsch.net}{Jesper@Dramsch.net}
\href{http://www.dramsch.net}{Dramsch.net}
\href{http://de.linkedin.com/in/thegeophysicist/en}{LinkedIn}
\section{languages}
\begin{tabular}{rr}German & \thinfont C2 
English& \thinfont C1
Spanish& \thinfont B1
French& \thinfont A2
Italian& \thinfont A1
Norwegian& \thinfont A1\end{tabular}
\section{software}
Promax \& SeismicUn*x
Matlab, Python \& \LaTeX
\section{geophysics}
seismic interpolation
sub-salt imaging
fresnelzone imaging
anisotropy
\end{aside}

%----------------------------------------------------------------------------------------
%	EDUCATION SECTION
%----------------------------------------------------------------------------------------


\section{education}

\begin{entrylist}
%------------------------------------------------
\entry
{2010--today}
{Master {\normalfont of Science, Geophysics}}
{University of Hamburg, Germany}
{\emph{Pre-stack Data Enhancement for Subsalt Imaging using the Partial Common Reflection Surface Stack} \\ This thesis will explore the improvement in seismic images in sub salt regimes, exploiting the redundancy of information in the midpoint as well as offset domain based on a Fresnel zone limited traveltime surface.}
%------------------------------------------------
\entry
{2007--2010}
{Bachelor {\normalfont of Science, Geophysics/Oceanography}}
{University of Hamburg, Germany}
{\emph{Trace interpolation using Partial CRS-Stacks} \\ This thesis evaluated the partial CRS-stack algorithm by interpolating seismic gaps that were artificially introduced in a data set.}
%------------------------------------------------
\entry
{2006--2007}
{Junior {\normalfont Studies, Geophysics/Oceanography}}
{University of Hamburg, Germany}
{A program for talented high school students, granting early access to university courses.}
\end{entrylist}

%----------------------------------------------------------------------------------------
%	WORK EXPERIENCE SECTION
%----------------------------------------------------------------------------------------

\section{experience}

\begin{entrylist}
%------------------------------------------------
\entry
{2013 (6m)}
{\href{http://www.uni-hamburg.de}{University of Hamburg}}
{Hamburg, Germany}
{\emph{Teaching Assistant -- Introduction to Geophysics} \\
Teaching a complementary lecture to the main introduction to geophysics lecture BSc course.}
%------------------------------------------------
\entry
{2012 (4m)}
{\href{http://www.slb.com}{Schlumberger}}
{Crawley, United Kingdom}
{\emph{Intern -- Engineering Seismic Interpolation} \\
Prototyping an automatic data-driven seismic interpolation method in MatLab.}
%------------------------------------------------
\entry
{2011 (2m)}
{\href{http://www.fugro.no/home/companies/fsi/downloads/fsi_oslo_company_profile}{Fugro Seismic Imaging}}
{Oslo, Norway}
{\emph{Intern -- R\&D Seismic Imaging} \\
Comparison of different near offset trace interpolation algorithms (i.e. CRS) to improve Surface Related Multiple Elimination (SRME) in 2D.}
%------------------------------------------------
\entry
{2010--2012}
{\href{http://www.uni-hamburg.de}{University of Hamburg}}
{Hamburg, Germany}
{\emph{Student Assistant -- Applied Seismics} \\
Assessing the accuracy of partial Common Reflection Surface (CRS) stacks for trace interpolation. Writing a primer on using the CRS software package.}
%------------------------------------------------
\entry
{2009 (3d)}
{FS Alkor}
{AL 339}
{\emph{University field work}\\
Including seismics, gravity, geomagnetics and echo-sounding.}
%------------------------------------------------
\entry
{2007 (2w)}
{\href{http://www.gfz-potsdam.de}{German Research Centre for Geosciences}}
{Potsdam, Germany}
{\emph{Intern -- Geophysical Deep Sounding} \\
Two week introduction to geophysical imaging depth imaging.}
%------------------------------------------------
\end{entrylist}

%----------------------------------------------------------------------------------------
%	AWARDS SECTION
%----------------------------------------------------------------------------------------

\section{awards}

\begin{entrylist}
%------------------------------------------------
\entry
{2012}
{Famelab Hamburg}
{\href{http://www.famelab-germany.de}{British Council}}
{2nd Prize for a very short public presentation on seismic imaging.}
%------------------------------------------------
\entry
{2011}
{\href{http://www.famelab-germany.de}{British Council}}
{\href{http://www.famelab-germany.de}{British Council}}
{1st Prize for a very short public presentation on seismology and geohazards.}
%------------------------------------------------
\end{entrylist}
%----------------------------------------------------------------------------------------
%	COMMUNICATION SKILLS SECTION
%----------------------------------------------------------------------------------------

\section{communication skills}

\begin{entrylist}
%------------------------------------------------
\entry
{2011-2012}
{Short Oral Presentation}
{\href{http://www.famelab-germany.de}{Famelab Germany}}
{Very short public presentations about research for my thesis.}
%------------------------------------------------
\entry
{2011}
{Poster Presentation}
{\href{http://earthdoc.eage.org/publication/publicationdetails/?publication=50683}{73\textsuperscript{rd} EAGE Vienna}}
{Poster presentation of the research conducted for my Bachelor thesis on trace interpolation.}
%------------------------------------------------
\entry
{2011}
{Technical Presentation}
{\href{http:/www.wit-consortium.de}{Wave Inversion Technology Meeting}}
{Technical presentation about my work involving partial CRS stacks and seismic interpolation.}
%------------------------------------------------
\entry
{2000--2007}
{Oral Presentation}
{\href{http://www.jugend-forscht.de/}{Jugend foscht}}
{Presentations on different self-chosen research projects including a written report in the geoscience section. "Jugend forscht" is the biggest youth science and technology competition in Europe.}
%------------------------------------------------
\end{entrylist}

%----------------------------------------------------------------------------------------
%	INTERESTS SECTION
%----------------------------------------------------------------------------------------
\section{voluntary activities}
\begin{entrylist}
\entry
{2011}
{Vice President}
{\href{http://www.geophysikstudenten.de/studentisches/gap}{Geophysical Activity Programme e.V.}}
{Major German geophysical student meeting and networking event. Including 10 major sponsors, 150 student participants, six organisation teams, several helping hands and a five figure budget.}
\entry
{2008--2011}
{Student representative}
{\href{http://geophysics.zmaw.de/index.php?id=fs}{University of Hamburg}}
{Involved in organising networking events, first semester introductory events and restructuring the representative organisation.}
\entry
{2007--today}
{Blood Donor}
{\href{http://www.blutspendediensthamburg.de}{Blutspendedienst Hamburg}}
{Consistent blood and plasma donor. \emph{(Over 25 blood and 20 plasma donations.)}}
\end{entrylist}
\section{interests}

\textbf{professional:} earth science, geophysics, geology, big data, seismic imaging, interpolation, science communication, data security, anisotropy \textbf{personal:} running, weight lifting, rock climbing, drums, saxophone, computer, role-playing games, strategic boardgames, organizing social events

%----------------------------------------------------------------------------------------
%	PUBLICATIONS SECTION
%----------------------------------------------------------------------------------------

\section{publications}

\printbibsection{article}{article in peer-reviewed journal} % Print all articles from the bibliography

\printbibsection{book}{books} % Print all books from the bibliography
\printbibsection{inbook}{book chapters} % Print all books chapters from the bibliography

\begin{refsection}
\nocite{*}
\printbibliography[type=inproceedings, title={international peer-reviewed conferences/proceedings}, heading=subbibliography]
\end{refsection}
\clearpage
\printbibsection{report}{research reports} % Print all research reports from the bibliography

\printbibsection{thesis}{theses} % Print all theses from the bibliography

\printbibsection{misc}{other publications} % Print all miscellaneous entries from the bibliography

%----------------------------------------------------------------------------------------

\section{references}

\refrow{\cvref{Dirk Gajewski}%
				{Department of Applied Seismics}%
				{Bundesstra\ss e 55}%
				{20146 Hamburg}%
				{Germany}%
				{dirk.gajewski@zmaw.de}%
				{+49 (0) 40 42838 2975}%
		}{%
		\cvref{Thomas Elboth}%
				{R\&D Project Manager at CGG}%
				{O.H. Bangs vei 70}%
				{1322 H\o vik, Oslo}%
				{Norway}%
				{thomas.elboth@cgg.com}%
				{+47 91 71 7680}%
		}
\refrow{\cvref{Claudia Vanelle}%
				{Department of Applied Seismics}%
				{Bundesstra\ss e 55}%
				{20146 Hamburg}%
				{Germany}%
				{claudia.vanelle@zmaw.de}%
				{+49 (0) 40 42838 5055}%
		}{%
		\cvref{Thomas Elboth}%
				{R\&D Project Manager at CGG}%
				{O.H. Bangs vei 70}%
				{1322 H\o vik, Oslo}%
				{Norway}%
				{thomas.elboth@cgg.com}%
				{+47 91 71 7680}%
		}


\hypersetup{urlcolor={lightgray}}
\footer{Jesper}{Dramsch}{\href{http://www.dramsch.net}{The Way of the Geophysicist}}

\end{document}